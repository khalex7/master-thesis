\section{Требования к серверным модулям и МУС}\label{tz}

Модули на микропроцессорах "Эльбрус" должны удовлетворять современным требованиям к серверам. В данном разделе указаны ограничения, наложенные на разработку модуля.

Микропроцессор, лежащий в основе модуля - "Эльбрус-8С2". Для данного микропроцессора еще не определены все режимы его работы, поэтому требуется реализовать функционал для наладки и тестирования модуля на этом микропроцессоре. Это второй восьмиядерный микропроцессор на архитектуре "Эльбрус" в линейке компании. От микропроцессора прошлого поколения "Эльбрус-8С" он отличается в первую очередь подсистемой памяти. В "Эльбрус-8С" была реализована память DDR3, в новом процессоре использован стандарт DDR4, что наложило на модуль новые требования. Кроме того, модули на данных процессорах отличаются системой питания микропроцессора. К новым модулям также были предъявлены требования по селективной пайке компонентов со сквозным креплением, интерфейсам I2C и JTAG.


Стандартом ATX регламентировано расположение слотов расширения. Среди возможных интерфейсов слотов указаны ISA, PCI, CNR и AGP. Но современные реалии требуют интерфейс PCI-Express (далее по тексту -- PCIe) в качестве основного для слотов расширения. Размещение слотов в данном случае регламентируется стандартом PCIe \cite{pcie_standart}.

В основе модуля лежат микропроцессор "Эльбрус-8С2" и контроллер периферийных интерфейсов КПИ-2 компании МЦСТ. Параметры и возможности этих микросхем определяют технические характеристики модуля. Основные требования к модулям E8C2-uATX/SE и E8C2-EATX описаны ниже. Технические характеристики модулей приведены в Приложении-А.

\subsection{Требования к однопроцессорному модулю E8C2-uATX/SE}

Модуль соответствует форм-фактору $\mu$ATX однопроцессорной платы. Данные серверные модули достаточно компактные и энергоэффективные, однако менее производительные, чем двух- и четырехпроцессорные. Потребляемая мощность модуля оценивается в 140Вт.


Габариты модуля выбраны в соответствии со стандартом $\mu$ATX.\cite{uatx_standart} Стандарт отличают компактные габариты (243,84мм $\times$ 243,84мм) платы вкупе с совместимостью с более крупными EATX и ATX, что позволяет использовать эти корпуса для модулей формфактора $\mu$ATX. На рисунке \ref{uatx_scheme} представлены основные размеры платы согласно стандарту:
\addimghere{uatx_scheme.png}{0.9}{Основные размеры платы форм-фактора $\mu$ATX}{uatx_scheme}

Требования к интерфейсам, основным микросхемам и параметрам модуля приведены в Приложении-А.

\subsection{Требования к двухпроцессорному модулю E8C2-EATX}

Модуль E8C2-EATX крупнее модуля форм-фактора MicroATX. На нем размещаются два микропроцессора "Эльбрус-8С2", по четыре канала памяти на каждом. 

Габариты модуля заданы 304,8 $\times$ 330,2мм. Потребляемая мощность модуля оценивается в 300Вт. Данный модуль предоставляет больше интерфейсов SATA, PCI Express, LAN и др.

Требования к интерфейсам, основным микросхемам и параметрам модуля приведены в Приложении-А.

\subsection{Требования к системе питания модулей E8C2-uATX/SE и E8C2-EATX}

Система питания модуля основана также на стандарте ATX и документации использованных микросхем. Потребляемая мощность микропроцессора -- 70Вт, источник напряжения питания ядра должен выдавать данную мощность.

Согласно стандарту ATX при подаче питания на плату подается начальное напряжение 5В (+5V\_SB).


По сигналу PS\_ON\# на модуль подаются номиналы +12V, +5V и +3V3, из которых с помощью преобразователей напряжения получены все номиналы модуля.
%Здесь указать про входные номиналы, сигнал PS ON, указать получаемые номиналы и порядок их включения. Источники разделить на три типа, ядро, остальные импульсники, линейники


\subsection{Требования к функционал для наладки модулей E8C2-uATX/SE и E8C2-EATX}
%Здесь про раздельный источник uncore, все тестпоинты, резисторы для съема тока, точки монтажа потенциометров и др.
Для проведения наладки модулей на новом микропроцессоре необходимо реализовать специальный функционал. В микропроцессоре Эльбрус-8С2 имеются следующие номиналы питания (за исключением тех, которым требуется фильтр, они получаются на плате из описанных):
 
\begin{itemize}
	\item +0V9\_CORE;
	\item +0V9\_UNCORE;
	\item +1V2;
	\item +1V8.
\end{itemize}

Микропроцессоры Эльбрус-8С2 имеют разделение питания периферии микропроцессора +0V9\_UNCORE на уровне корпуса на питание подсистемы памяти и соответствующих PLL +0V9\_MC и питание подсистемы линков и соответствующих им PLL +0V9\_LINK. В существующих модулях на данном 
микропроцессоре было подозрение, что область допустимого напряжения этих двух доменов питания не пересекается. Для установления области допустимого напряжения требуется разместить два отдельных преобразователя напряжений на данные домены питания. На преобразователях требуется реализовать возможность менять выходное напряжение запаиванием контактных площадок, а также контактные площадки для установки потенциометра, предоставляющего возможность программным образом менять выходное напряжение и тем самым проводить тестирование. 

Также на всех основных источниках требуется установить контактные площадки для съема напряжения. На основном питании ядра микропроцессора, которое потребляет наибольшую мощность, есть возможность подключиться сразу к трем точкам:\label{tp-of-power}

\begin{table}[H]
	\caption{Точки съема напряжения питания ядра с микропроцессора}
	\begin{tabular}{|c|c|}
		\hline Индекс & Местоположение  \\
		\hline А0\_AP & Центр SIC\_global. Слой источников.   \\
		\hline A0\_M1 & Центр SIC\_global. Слой потребителей.  \\
		\hline А1\_AP & Зона кэша L2 между процессорными ядрами  \textnumero0 и  \textnumero1.  \\
		\hline A1\_M1 & Зона кэша L2 между процессорными ядрами  \textnumero0 и  \textnumero1.  \\
		\hline А2\_AP & Зона процессорного ядра  \textnumero4. Слой источников. \\
		\hline A2\_M1 & Зона процессорного ядра  \textnumero4. Слой потребителей.  \\
		\hline 
	\end{tabular}
\end{table}
\addimghere{points_of_voltage.png}{0.45}{Точки съема напряжения питания ядра с микропроцессора}{tp_core}

Кроме данных точек на контактные площадки выведены точки возле преобразователя напряжения и по одному пину питания микропроцессора.

На тестовые точки других номиналов также выведены точки возле преобразователя напряжения и по одному пину питания микропроцессора. Это сделано для следующих номиналов:

\begin{itemize}
	\item +0V0\_MC;
	\item +0V9\_LINK;
	\item +1V2.
\end{itemize}

\subsection{Требования к модулю управления системой МУС-А}

Модули управления системой широко распространены на рынке серверов. Все больше возникает необходимость их использования в серверных кластерах, так как они позволяют достаточно быстро определять и устранять неполадки в сервере даже при зависшей операционной системе или включать, выключать, перезагружать с удаленной машины, не находясь при этом непосредственно у сервера.

За основу модуля взята система-на-кристалле Aspeed AST2400. Основные характеристики микросхемы:

\begin{itemize}
	\item ядро ARM9, 400MHz;
	\item память DDR2/DDR3;
	\item загрузка по SPI;
	\item поддержка iKVM;
	\item 2D видео-адаптер (только в AST2400);
	\item потребляемая мощность 0,5-1,7Вт;
	\item размеры 19x19 мм.
\end{itemize}

В модуле требуется реализовать следующие интерфейсы, которые позволяют удаленно управлять системой:

\begin{itemize}
	\item TIA-232-F, 2шт.;
	\item LAN 10/100, 2шт.;
	\item I2C, 5шт.;
	\item IPMB, 2шт.;
	\item SPI флеш-памяти МУС;
	\item SPI флеш-памяти бута и NVRAM;
	\item JTAG, 1шт.;
	\item GPIO 15шт.
\end{itemize}


Модули управления системой могут быть поставлены как отдельной платой, так и быть интегрированы в серверный модуль с использованием специальных микросхем. МУС фактически являются самостоятельной вычислительной машиной, на которую установлено собственное ОПО и СПО, позволяющее пользоваться ее функционалом. 

В серверных модулях предприятия реализована поддержка МУС фирмы Pigeon Point PPMM-700R в форм-факторе DDR3 SODIMM. Модуль МУС-А должен быть совместим с соединителем и интерфейсами для МУС Pigeon Point.   

\clearpage
