

%Список ненумерованный
\begin{itemize}
	\item ;
	\item .
\end{itemize}

%Список буквенный
\begin{enumerate}
	\item ;
	\item .
\end{enumerate}


%Цитата
\cite{dc-tier}.

%Процент
\%

%Рисунок в начале страницы
Архитектура полной виртуализации представлена на рис. \ref{full-virt}.
\addimg{full_virt}{0.35}{Архитектура полной виртуализации}{full-virt}
%Рисунок прямо после текста
Архитектура процесса эмуляции представлена на рис. \ref{emul}.
\addimghere{emulation}{0.35}{Эмуляция аппаратных средств}{emul}

%Таблица
Матрица инциденций функций системы и функций назначения подсистем приведена в таблице \ref{inc-matrix}.
\begin{table}[H]
	\caption{Матрица инциденций}\label{inc-matrix}
	\begin{tabular}{|c|c|c|c|c|c|}
		\hline \multirow{2}{*}{Функции} & \multicolumn{4}{|c|}{Подсистемы} & \multirow{2}{*}{Виртуальная инфраструктура} \\
		\cline{2-5} & 1 & 2 & 3 & 4 & \\
		\hline Ф1 & + & & & & + \\
		\hline Ф2 & + & & & & + \\
		\hline Ф3 & & + & & & + \\
		\hline Ф4 & & + & & & + \\
		\hline Ф5 & & & + & + & + \\
		\hline 
	\end{tabular}
\end{table}

\begin{table}[H]
	\caption{Точки съема напряжения питания ядра с микропроцессора}
	\begin{tabular}{|c|c|}
		\hline Ф1 & +  \\
		\hline Ф2 & +  \\
		\hline Ф3 & + \\
		\hline Ф4 & + \\
		\hline Ф5 & + \\
		\hline 
	\end{tabular}
\end{table}

%Код
\begin{lstlisting}
# cat /etc/ntp.conf | grep ^server | awk '{print $2}'
0.debian.pool.ntp.org
1.debian.pool.ntp.org
2.debian.pool.ntp.org
3.debian.pool.ntp.org
\end{lstlisting}