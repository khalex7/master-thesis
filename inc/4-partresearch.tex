\section{Исследование компонент}

Проведен анализ компонентной базы модулей. Выбор компонентов описан в данном разделе.

\subsection{Система питания}

Для системы питания нужны три типа преобразователей напряжения.

\begin{itemize}
	\item линейный преобразователь напряжения (до 2А);
	\item импульсный однофазный преобразователь напряжения (2-40А);
	\item импульсный многофазный преобразователь напряжения (от 40А).
\end{itemize}

Линейные преобразователи нужны для маломощных потребителей, как правило, для одиночных микросхем с уникальным на плате напряжением. Также они применены для номиналов стартового питания, когда машина выключена, но некоторые маломощные микросхемы должны работать для организации включения и управления платой. Однофазные импульсные преобразователи дают все рабочие номиналы микросхем и соединителей. Многофазный импульсный преобразователь дает основной номинал питания ядра микропроцессора, который требует большую мощность вкупе со стабильностью номинала.

\subsubsection{Линейные преобразователи напряжения}\label{linear_choose}

Линейные преобразователи выбирались двух типов. Менее мощный и более дешевый до 500мА и более мощный ~ 2А.

Были рассмотрены следующие преобразователи:
\begin{itemize}
	\item AP7331-WG-7 -- Diodes Incorporated;
	\item MAX8527EUD+ -- Maxim Integrated;
	\item MIC69153 -- Microchip Technology.
\end{itemize}

%Таблица
Сравнение данных компонентов приведено в таблице \ref{linear-power}.
\begin{table}[H]
	\caption{Линейные источники напряжения}\label{linear-power}
	\begin{tabular}{|l|l|l|l|}
		\hline Номер микросхемы & 
		\begin{sideways} AP7331-WG-7 \end{sideways} & 
		\begin{sideways} MAX8527EUD+ \end{sideways} &
		\begin{sideways} MIC69153YML \end{sideways}   \\
		\hline Максимальный выходной ток, мА & 300 & 2000 & 1500 \\
		\hline Входное напряжение, В & 2-6 & 1,425-3,6 & 1,65-5,5 \\
		\hline Выходное напряжение, В & 0,8-5 & 0,5-(Vin-0,2) & 0,5-5,5 \\
		\hline Размеры, мм и корпус & 3x3, SOT25 & 5x6,4, TSSOP & 3x3, DFN  \\
		\hline Цена, \$ & 0.15 & 2,2 & 1,5 \\
		\hline 
	\end{tabular}
\end{table}

Выбраны первый преобразователь как самый дешевый для номиналов, не требующих тока больше 300мА, и второй для токов до 2А.

\subsubsection{Импульсные однофазные источники}

Было рассмотрено большое число импульсных источников для всех основных номиналов. Потребности в мощности для разных номиналов отличаются, поэтому требовалось подобрать различные импульсные источники с максимальным током в 3, 8-12 и 20-40 Ампер. Среди рассмотреных и примененных в различных проектах источников напряжения для рассмотрения были выбраны следующие:
 
\begin{itemize}
	\item PDT012A0X3-SRZ;
	\item MDT040A0X3-SRPHZ;
	\item UDT020A0X3-SRZ;
	\item PDT003A0X3-SRZ;
	\item IR3843AMTRPBF;
	\item TPS53318DQP;
	\item IR3899AMTRPBF;
	\item TPS543C20RVFT.
\end{itemize}

%Таблица
Сравнение данных компонентов приведено в таблице \ref{impulse-onephase-power}.
\begin{table}[H]
	\caption{Линейные источники напряжения}\label{impulse-onephase-power}
	\begin{tabular}{|p{3,1cm}|l|l|l|l|l|l|l|l|}
		\hline Номер микросхемы & 
	\begin{sideways} PDT012A0X3-SRZ -- General Elecrics \end{sideways} & 
		\begin{sideways} MDT040A0X3-SRPHZ -- General Elecrics \end{sideways} &
		\begin{sideways} UDT020A0X3-SRZ -- General Elecrics \end{sideways} &
		\begin{sideways} PDT003A0X3-SRZ -- General Elecrics \end{sideways} &
		\begin{sideways} IR3843AMTRPBF -- Infineon \end{sideways} &
		\begin{sideways} IR3899MTRPBF -- Infineon \end{sideways} &
		\begin{sideways} TPS53318DQP -- Texas Instruments \end{sideways} &
		\begin{sideways} TPS543C20RVFT -- Texas Instruments \end{sideways}   \\
		\hline Максимальный выходной ток, А & 12 & 40 & 20 & 3 & 3 & 9 & 8 & 40 \\
		\hline Входное напряжение, В & 3-14,4 & 4,5-14,4 & 3-14,4 & 3-14,4 & 1,5-21 & 1-21 & 1,5-22 & 4-14 \\
		\hline Выходное напряжение, В & 0,6-5,5 & 0,6-2 & 0,6-5,5 & 0,6-5,5 & 0,7-19 & 0,5-18 & 0,6-5,5 & 0,6-5,5 \\
		\hline Встроенная индуктивность& + & + & + & + & - & - & - & - \\
		\hline Шина PMBus & + & + & + & - & - & - & - & - \\
		\hline Размеры, мм & 12x12 & 33x14 & 20x11 & 12x12 & 5x6 & 4x5 & 5x6 & 5x7 \\
		\hline Цена, \$ & 10 & 25 & 12 & 5 & 1,5 & 2 & 4 & 7 \\
		\hline 
	\end{tabular}
\end{table}

Необходимо отметить, что у преобразователей, не имеющих встроенной индуктивности, необходимо отдельно закупать и ставить катушку индуктивности, что повлечет увеличение стоимости и занимаемого на плате места. На выбор компонентов оказал влияние наладочный функционал преобразователей. Для данных плат наличие шины PMBus было признано решающим по сравнению с ценой и занимаемым местом, поэтому для данных модулей выбраны преобразователи 3А, 12А, 20А и 40А General Electric, остальные рассматриваются для следующих модулей и серийного производства. 

В результате выбрана оптимальная компонентная база модулей.

\clearpage
