\section{Размещение компонентов, трассировка плат, проведение DRC генерирование файлов и создание КД}

Рассмотрены особенности разработки топологии плат, поиска ошибок и создание файлов для КД и производства. Данные действия выполняются в ПО фирмы Mentor Graphics -- Expedition PCB.

\subsection{Размещение компонентов и трассировка плат}

\subsubsection{Форм-фактор и структура}
Модули построены на основе модулей соответствующих форм-факторов на основе микропроцессора "Эльбрус-8С" (E8C-uATX/SE и E8C-EATX). В данных модулях удален процессорный блок. Общая схема компонентов модулей форм-фактора uATX представлена на рисунке \ref{uatx-sch}:
\addimghere{uatx_sch.png}{0.8}{Расположение}{uatx-sch}

На рисунке изображены: 1 -- процессорный блок; 2 -- блок соединителей задней панели; 3 -- соединитель МУС; 4 -- контроллер периферийных интерфейсов КПИ-2.

В обоих модулях выбрана структура из 14 слоев, из которых 6 сигнальных. Структура представлена на рис. \ref{eatx-stackup}:
\addimghere{eatx-stackup.png}{0.85}{Струкутура платы двухпроцессорного модуля E8C2-EATX}{eatx-stackup}

В однопроцессорном модуле использована аналогичная структура. Расчетное волновое сопротивление 50-55Ом для одиночных сигналов, 92-100Ом для дифференциальных пар - межпроцессорных линков и линков ввода-вывода, 82,5 - 90Ом для остальных диффпар.

\subsubsection{Процессорный блок}

В обоих модулях разработан процессорный блок. В uATX компоненты процессорного блока размещались на самой плате. В EATX отдельно от модуля разработан типовой процессорный блок, который установлен на модуль в двух экземплярах. Требовалось ограничить типовой блок размерами блока на процессоре "Эльбрус-8С". Типовой блок представлен на рис. \ref{reusable-block}

В составе типового блока установлены: микропроцессор, 4 соединителя модулей памяти, всего по 1 модулю на канал, система питания микропроцессора, генераторы синхросигнала микропроцессора. Произведена трассировка процессорного блока. Блок установлен на модуль.

\addimghere{reusable-block.png}{0.5}{Типовой процессорный блок м.п. "Эльбрус-8С2"}{reusable-block}

Установлены точки подключения внешнего генератора синхросигнала для исследования подачи синхросигнала разных частот на микропроцессор. Точки съема представлены на рисунке \ref{tp-of-clk}:

\addimghere{tp-of-clk.png}{0.9}{Точки подключения синхросигнала}{tp-of-clk}

Здесь зеленым отмечены диффпара синхросигнала и контактные площадки генератора синхросигнала на нижнем слое. Синим отмечены контактные площадки и отверстия синхросигнала. 

Испытания показали, что проблемы высокочастотных сигналов возникают из-за нарушений в линии (стабы и спейсинги) и помехи в линии (при переходе на другой слой).
 
В предыдущих модулях на микропроцессоре "Эльбрус-8С2" линии памяти DDR4 сопровождались только землей. Так как в соединителе памяти некоторые линии сопровождены только землей, а некоторые только питанием, в модуле E8C2-uATX/SE решено сопроводить линии памяти и землей, и питанием. Получившаяся структура платы в области памяти показана на рис. \ref{mc-struct}:
\addimghere{mc-struct.png}{0.5}{Структура платы E8C2-uATX/SE в области памяти}{mc-struct}

В модуле E8C2-EATX линии памяти DDR4, оттрассированные на внутренних слоях, сопровождаются полигонами питания памяти +1V2 и земли GND на всем протяжении трасс от микропроцессора до соединителей памяти. Так как в соединителе памяти линии данных сопровождаются только землей, а линии адреса только питанием, возникает разрыв в сопровождении данных линий. Для уменьшения влияния данного разрыва решено установить разрыв возле соединителей памяти, как представлено на рис. \ref{mc-planes}:
\addimghere{mc-planes.png}{0.75}{Разрыв полигонов сопровождения памяти}{mc-planes}

Перед планками памяти осуществляется разрыв, на котором установлены высокочастотные конденсаторы (зеленые и серые контактные площадки). Под планками памяти (после разрыва) линии адреса сопровождаются только полигонами питания (лиловый цвет), а линии данных только землей (темно-серый цвет). Таким образом уменьшено влияние разрыва.

\subsubsection{Полигоны питания}

В модулях были начерчены полигоны питания на слоях питания. Средствами САПР фирмы Mentor Graphics -- Hyperlynx произведен анализ полигонов и получены данные по падению напряжения на полигонах. На рис. \ref{0v9-core} представлено графическая визуализация падения питания ядра микропроцессора в модуле E8C2-uATX/SE, проведенного на трех слоях: 1, 6 и 7:
\addimghere{0V9_CORE.jpg}{0.8}{Анализ падения напряжения питания ядра}{0v9-core} 

Получено максимальное падение напряжения в 15,6мВ, что является допустимым значением.

\subsubsection{Интерфейсы}

В интерфейсах PCI Express, USB, IOLink, IPLink, представленных дифференциальными парами, для устранения влияния помех, добавлено сопровождение отверстий данных интерфейсов отверстиями земли. Пример реализации представлен на рисунке \ref{diffpair-via}:

\addimghere{diffpair-via.png}{0.8}{Сопровождение отверстий интерфейсов отверстиями земли}{diffpair-via}

Здесь серым указаны отверстия земли, красным -- отверстия диффпар и трассы диффпар на одном слое, белым -- на другом слое.

\subsubsection{Точки съема напряжения}

На модулях установлены точки съема напряжений питания ядра с разных точек и других номиналов, как описано в главе \ref{tp-of-power} на стр. \pageref{tp-of-power}. Точки съема на модуле E8C2-uATX/SE представлены на рис. \ref{tp-power}:

\addimghere{tp-of-power.png}{0.8}{Точки съема напряжения}{tp-power}

\subsection{Проведение DRC}

Настроены параметры поиска ошибок DRC. В результате проведения DRC исправлены нарушения технологических норм по размещению на расстоянии меньше допустимого: компонентов относительно друг друга, трасс относительно друг друга и контактных площадок, расположению компонентов в недопустимых зонах, короткого замыкания и др.

Сверены электрическая схема, размещение и трассировка с документацией на компоненты. 

\subsection{Генерирование файлов и создание КД}

Сгенерированы следующие файлы, назначение которых указано в разделе о маршруте проектирования \ref{output-files} на стр. \pageref{output-files}
\begin{itemize}
	\item BOM-файл;
	\item Gerber-файл;
	\item DXF-файлы;
	\item файлы NCDrill;
	\item vb\_ais и gencad.cad.
\end{itemize}

Данные файлы отправлены на завод для производства платы модуля.

Созданная на основе этих файлов и электрической схемы конструкторская документация проверена и использована для закупки и монтажа компонентов.


\clearpage
