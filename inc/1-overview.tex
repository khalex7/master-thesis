\section{Цель работы и задачи} \label{literature}

\subsection{Цель работы}

Целью работы является разработка однопроцессорного и двухпроцессорного серверных модулей на микропроцессоре "Эльбрус-8С2" и модуля управления системой МУС-А, входящего в состав данных серверов.

\subsection{Требования к работе}

Предъявлены следующие требования к работе:

\begin{itemize}
	\item разработать серверные модули: однопроцессорный E8C2-uATX/SE - форм-фактор Micro ATX и двухпроцессорный E8C2-EATX - форм-фактор Extended ATX ;
	\item реализовать на данных модулях функционал для наладки;
	\item разработать модуль управления системой на основе микросхемы Aspeed AST2400 в форм-факторе DDR3 SODIMM; разработанный МУС должен быть совместим с МУС Pigeon Point PPMM-700R;
	\item адаптировать программные средства для МУС.
\end{itemize}

\subsection{Задачи}

В ходе анализа требований были исследованы существующие решения и поставлены следующие задачи:

\begin{enumerate}
	\item Формирование и учет основных ограничений – выбор корпуса, расположение компонентов, типов корпусов, выбор электронно-компонентной базы, технологические ограничения (технологии монтажа, тестирования).
	\item Выполнение проектирования согласно маршруту с учетом ограничений: 
 \begin{itemize}
 	\item разработка функциональной схемы: определение основных компонентов, всех интерфейсов, номиналов питания, порядка включения системы питания, синхросигналов, карты служебных шин;
 	\item разработка электрической схемы: определение компонентов на плате, расчет всех значений резисторов,	конденсаторов, индуктивностей согласно документации на компоненты, подключение всех пинов компонентов к линиям и техническим символам;
 	\item размещение и трассировка: определение границ платы, структуры платы, расположения основных компонентов согласно стандартам, чертеж полигонов питания на слоях питания. Размещение остальных компонентов согласно документации микросхем и требованиям селективной пайки. Трассировка высокочастотных интерфейсов согласно стандартам, затем трассировка остальных линий, размещение шелкографии на плате;
 	\item проверка платы, проведение DRC, исправление ошибок;
 	\item генерирование финальных файлов для отправки на завод;
 	\item подготовка файлов для создания конструкторской документации;
 	\item тестирование и наладка.
 \end{itemize}
	\item Участие в создании конструкторской, программной и технологической документации.
\end{enumerate}

\clearpage