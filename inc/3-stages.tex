\section{Этапы разработки модулей}

С учетом ограничений, заданных выше, выполнены все пункты маршрута проектирования, включающего следующие этапы, описанные в данном разделе:

%Список ненумерованный
\begin{itemize}
	\item разработка функциональной схемы модуля;
	\item исследование и выбор компонентов;
	\item разработка электрической схемы модуля;
	\item размещение компонентов и трассировка;
	\item проведение DRC (design rules check) и исправление ошибок;
	\item генерирование финальных файлов для отправки на завод;
	\item создание конструкторской, программной и технологической документации модуля;
	\item запуск в производство.
\end{itemize}

Данный этап применим как для серверных модулей, так и для модуля управления системой.

На этапе разработки функциональной схемы определены основные интерфейсы, компоненты и способ их взаимодействия. Создана графическая модель модуля, описывающая и связывающая все функциональные узлы модуля без описания реализации этих узлов. Так как модель не имеет собственных обозначений, использованы термины принципиальной электрической схемы. В схеме представлены следующие страницы:

\begin{itemize}
	\item схема модуля, представлены основные компоненты интерфейсы;
	\item схема системы синхронизации модуля, представлены все сигналы синхронизации микросхем модуля;
	\item схема системы питания, представлены все номиналы напряжений модуля, а также все преобразователи напряжения;
	\item схема включения модуля, представлена диаграмма сигналов;
	\item (опционально) схема модуля управления системой (МУС), аппаратно-программного модуля доверенной загрузки (АПМДЗ) и контроллера периферийных интерфейсов (КПИ-2).
\end{itemize}

При разработке принципиальной электрической схемы определены все компоненты, которые будут установлены на плату. Произведен рассчет значений всех резисторов, конденсаторов, индуктивностей и других компонентов, подключены пины всех компонентов друг к другу и к служебным символам.

При размещении и трассировке определены границы платы и ее структура. Расположены основные компоненты согласно стандарту форм-фактора. Затем расставлены остальные компоненты, произведена трассировка и определение полигонов питания на слоях питания.

При проведении DRC методами САПР проведен поиск технологических ошибок, таких как короткое замыкание. Проведена проверка электрической схемы, размещения и трассировки на соответствие документации и поставленным задачам. Все найденные ошибки устранены.

Сгенерированы следующие файлы: \label{output-files}
\begin{itemize}
	\item BOM-файл со списком компонентов, используется для создания спецификации;
	\item Gerber-файлы, предоставляют собой послойное описание платы для изготовления фотошаблона на фабрике;
	\item DXF-файлы, чертежи верхнего и нижнего слоя для создания сборочного чертежа;
	\item файлы NCDrill, информация о типе и координатах металлизированных и неметаллизированных отверстий платы;
	\item vb\_ais и gencad.cad, информация о расположении компонентов.
\end{itemize}

Полученные файлы, схемы и проект переданы для создания конструкторской документации. В результате разработки КД создана спецификация, согласно которой произведены закупки компонентов. Файлы также отправлены на завод для создания печатных плат. После получения платы и компонентов осуществлен монтаж компонентов, наладка и тестирование.

\clearpage
