\appsection{Перечень условных обозначений} \label{perechen}
\vspace{\baselineskip}


\begin{itemize}
	\item E8C2-uATX/SE -- однопроцессорный модуль на микропроцессоре "Эльбрус-8С2" в форм-факторе MicroATX;
	\item E8C2-EATX -- двухпроцессорный модуль на микропроцессоре "Эльбрус-8С2" в форм-факторе ExtendedATX;
	\item МУС -- модуль управления системой;
	\item МУС-А -- МУС, рассматриваемый в данной работе;
	\item системный модуль -- модуль, в который установлен рассматриваемый МУС
	\item DRC -- design rule checking, проверка правил проектирования платы;
	\item FRUID -- память типа EEPROM, в которую система;
	\item ПО -программное обеспечение;
	\item ОПО и СПО -- общее и специальное программное обеспечение;
	\item КПИ-2 -- контроллер периферийных интерфейсов фирмы МЦСТ;
	\item АПМДЗ -- аппаратно-программный модуль доверенной загрузки;
	\item САПР -- средство автоматического проектирования;
	\item КД -- конструкторская документация;
	\item ПД -- программная документация;
	\item ТД -- технологическая документация;
	\item ШИМ -- широтно-импульсная модуляция;
	\item Bitbake -- программа сборки программных средств;
	\item ОС -- операционная система.
\end{itemize}
 


\clearpage

\anonsection{Введение}

\paragraph{Серверные модули}

Серверные модули имеют в первую очередь сервисное предназначение. Различные цифровые устройства и вычислительная техника подключается к серверам для использования их вычислительных мощностей или другого функционала сервера для своих задач. \cite{windows_server_app} 
Кроме того, серверные модули предназначены для работы в связке с другими серверами - кластере, системе хранения данных и др. Они должны быть легко доступными для контроля и управления в составе такой системы.

Серверы предназначены для выполнения многих задач, например таких как:

\begin{itemize}
	\item файловое хранилище;
	\item организация печати;
	\item веб-службы;
	\item организация удаленного доступа;
	\item исполнение программ;
	\item сервер электронной почты;
	\item хранение базы данных.
\end{itemize} \cite{windows_server_app}

Перечисленные функции серверов являются основными. Для поддержания всего функционала для модуля сервера необходима возможность подключения накопительной памяти большого объема, оперативная память большого объема, производительный микропроцессор или несколько, широкий набор периферийных интерфейсов, включая сетевые. Для создания отечественных аналогов зарубежных серверов в рамках импортозамещения разрабатываются серверные модули на основе микропроцессоров линейки "Эльбрус", позволяющие выполнять данные задачи.

За основу серверного модуля предлагается взять новую разработку компании МЦСТ -- восьмиядерный микропроцессор "Эльбрус-8С2". Особенностью разработки модулей под данный микропроцессор является тот факт, что финальная версия микропроцессора "Эльбрус-8С2" еще не была выпущена во время разработки модулей, в связи с чем отсутствуют данные об условиях работы нового микропроцессора. В разрабатываемых модулях требуется предоставить возможность определить эти условия. 

\paragraph{Модули управления системой}

Модули управления системой (МУС) -- модули, интегрированные в плату сервера или подключаемые к ней. МУС имеют возможность проводить с модулем сервера следующие действия: \cite{mus-c}

\begin{itemize}
	\item включать, выключать, перезагружать модуль сервера;
	\item предоставлять доступ к последовательному порту системы;
	\item считывать и записывать информацию во флеш-память FRUID;
	\item считывать показания датчиков температуры, преобразователей напряжения, управлять вентиляторами;
	\item записывать системные логи сервера;
	\item производить другие действия согласно спецификации IPMI 2.0.
\end{itemize}

Данный функционал реализует управление серверным модулем, используя сеть МУС. МУС предоставляет возможность решать проблемы с нештатной работой любого сервера из состава кластера, перенастраивать его и совершать иные действия с удаленной машины.

МУС входят в комплект поставки серверов. На предприятии в составе серверных модулей поставляются МУС фирмы Pigeon Point PPMM-700R. Рассмотренный в данной работе моодуль МУС-А призван заменить импортный аналог более дешевым модулем в комплектациях серверов на микропроцессорах "Эльбрус".
 
Подключение к МУС осуществляется через последовательный порт или по выделенной локальной сети МУС.После подключения пользователь получает возможность выполнять описанные действия из командной оболочки операционной системы МУС.

\clearpage
