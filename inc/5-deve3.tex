\section{Разработка функциональной и электрической схем}

В данном разделе рассмотрены этапы разработки функциональной и электрической схем.

За основу взяты модули на основе микропроцессора "Эльбрус-8С". В данных модулях удален процессорный блок, разработан блок на основе "Эльбрус-8С2", после чего модули были доработаны.

\subsection{Функциональная схема}
%Здесь про все страницы Э2
Разработана схема модулей. В ней определены все интерфейсы, соединители, основные микросхемы, синхросигналы и номиналы питания. Схема представлена на рис. \ref{func-uatx}:
\addimghere{func-uatx.png}{0.8}{Схема однопроцессорного модуля E8C2-uATX/SE}{func-uatx}


\subsection{Электрическая схема}

Для разработки электрической схемы использовано программное обеспечение фирмы Mentor Graphics -- Design Capture. 

\subsubsection{Разработка типового блока микропроцессора Эльбрус-8С2}

Типовой блок микропроцессора "Эльбрус-8С2" состоит из самого микропроцессора, модулей памяти, систем питания и синхронизации.

Микропроцессор имеет 4 канала памяти DDR4, на которых доступно размещать по две планки памяти на канал. С учетом габаритов модулей выбрано 4 модуля памяти на один микропроцессор. Соединители модулей памяти подключены согласно стандарту DDR4. Пины адреса I2C шины модулей памяти утянуты в соответствии с адресом модуля. Адреса указаны на итоговой карте I2C устройств в пункте \ref{i2c-map} на стр. \pageref{i2c-map}.

Разработка системы питания рассмотрена в следующем пункте \ref{powersys}.

Система синхронизации модуля состоит из генераторов синхросигнала LVDS. Выбраны генераторы с частотой 125МГц XLL736125.000000I фирмы IDT с возможностью замены на 500МГц для проведения исследований микропроцессора с данной частотой.

\subsubsection{Разработка системы питания}\label{powersys}
%Привести все расчеты СП
Разработка системы питания заключается в расчете показателей  

Схема задания выходного напряжения преобразователя AP7331-WG-7 представлена на рисунке \ref{ap7331-vout-setup} \cite{ap7331-c}:
\addimghere{ap7331-vout-setup.png}{0.35}{Схема преобразователя AP7331-WG-7}{ap7331-vout-setup}

Vin = 5В (+5V\_SB); Vref = 0,4В;

Номинал +1V0\_SUS: 

R2=1кОм, расчет сопротивления для искомого напряжения:
\begin{equation}
R_{1} = R_{2} \times \left(\frac{V_{OUT}}{V_{REF}} - 1 \right) = 1,5\text{кОм}
\end{equation}

Взят резистор R1=1,5кОм; получено выходное напряжение:

\begin{equation}
V_{OUT} = V_{REF} \times \left(1 + \frac{R_{1}}{R_{2}} \right) = 1\text{В}
\end{equation}

Номинал +1V8\_SUS:

R2=1кОм, расчет сопротивления для искомого напряжения:
\begin{equation}
R_{1} = R_{2} \times \left(\frac{V_{OUT}}{V_{REF}} - 1 \right) = 3,5\text{кОм}
\end{equation}

Взят резистор R1=3,48; получено выходное напряжение:

\begin{equation}
V_{OUT} = V_{REF} \times \left(1 + \frac{R_{1}}{R_{2}} \right) = 1,79\text{В}
\end{equation}

Номинал +2V5\_SUS:

R2=1кОм, расчет сопротивления для искомого напряжения:

\begin{equation}
R_{1} = R_{2} \times \left(\frac{V_{OUT}}{V_{REF}} - 1 \right) = 5,25\text{кОм}
\end{equation}

Взят резистор R1=5,23; получено выходное напряжение:

\begin{equation}
V_{OUT} = V_{REF} \times \left(1 + \frac{R_{1}}{R_{2}} \right) = 2,49\text{В}
\end{equation}


Схема задания выходного напряжения преобразователя MAX8527EUD+ представлена на рисунке \ref{max8527-vout-setup}:
\addimghere{max8527-vout-setup.png}{0.35}{Схема преобразователя MAX8527EUD+}{max8527-vout-setup} \cite{max8527-c}

Vfb=0,5В; 

Номинал +1V8:

R2=2,74кОм, расчет сопротивления для искомого напряжения:

\begin{equation}
R_{1} = R_{2} \times \left(\frac{V_{OUT}}{V_{FB}} - 1 \right) = 7,12\text{кОм}
\end{equation}

Взят резистор R1=7,15; получено выходное напряжение:

\begin{equation}
V_{OUT} = V_{FB} \times \left(1 + \frac{R_{1}}{R_{2}} \right) = 1,80\text{В}
\end{equation}

Импульсные:

PDT012A0X3-SRZ, UDT020A0X3-SRZ, MDT040A0X3-SRPHZ

Схема задания выходного напряжения преобразователя PDT012A0X3-SRZ представлена на рисунке \ref{pdt012-vout-setup}  \cite{pdt012-c}:
\addimghere{pdt012-vout-setup.png}{0.35}{Схема преобразователя PDT012A0X3-SRZ}{pdt012-vout-setup}

Номинал +1V0:

Расчет сопротивления для искомого напряжения:

\begin{equation}
R_{TRIM} = \frac{12}{V_{OUT}-0,6} = 30\text{кОм}
\end{equation}

Взят резистор R1=30,1кОм; получено выходное напряжение:

\begin{equation}
V_{OUT} = 0,6 + \frac{12}{R_{TRIM}} = 1,00\text{В}
\end{equation}

Номинал +1V2:

Расчет сопротивления для искомого напряжения:

\begin{equation}
R_{TRIM} = \frac{12}{V_{OUT}-0,6} = 20\text{кОм}
\end{equation}

Взят резистор R1=20кОм; получено выходное напряжение:

\begin{equation}
V_{OUT} = 0,6 + \frac{12}{R_{TRIM}} = 1,20\text{В}
\end{equation}


Номинал +2V5:

Расчет сопротивления для искомого напряжения:

\begin{equation}
R_{TRIM} = \frac{12}{V_{OUT}-0,6} = 6,32\text{кОм}
\end{equation}

Взят резистор R1=6,34кОм; получено выходное напряжение:

\begin{equation}
V_{OUT} = 0,6 + \frac{12}{R_{TRIM}} = 2,49\text{В}
\end{equation}

Номинал +2V5:

Расчет сопротивления для искомого напряжения:

\begin{equation}
R_{TRIM} = \frac{12}{V_{OUT}-0,6} = 6,32\text{кОм}
\end{equation}

Взят резистор R1=6,34кОм; получено выходное напряжение:

\begin{equation}
V_{OUT} = 0,6 + \frac{12}{R_{TRIM}} = 2,49\text{В}
\end{equation}

Номиналы +3V3\_SUS и +3V3\_1:

Расчет сопротивления для искомого напряжения:

\begin{equation}
R_{TRIM} = \frac{12}{V_{OUT}-0,6} = 4,44\text{кОм}
\end{equation}

Взят резистор R1=4,42кОм; получено выходное напряжение:

\begin{equation}
V_{OUT} = 0,6 + \frac{12}{R_{TRIM}} = 2,49\text{В}
\end{equation}

Номиналы +0V9\_SUS и +3V3\_1:

Расчет сопротивления для искомого напряжения 0,950В:

\begin{equation}
R_{TRIM} = \frac{12}{V_{OUT}-0,6} = 34,29\text{кОм}
\end{equation}


Установлен массив резисторов для возможного увеличения и уменьшения напряжения путем запаивания контактных площадок.
Начальное сопротивление массива R0=34,13кОм; получено выходное напряжение:

\begin{equation}
V_{OUT} = 0,6 + \frac{12}{R_{TRIM}} = 0,952\text{В}
\end{equation}


Многофазный преобразователь напряжения питания ядра на основе контроллера TPS40140. \cite{tps40140-c}

Максимальный выходной ток для одной фазы 20А. Максимальный ток микросхемы "Эльбрус-8С2" 120А. Исходя из этих значений установлено число фаз -- 6.

Одна фаза преобразователя схематически представляет собой модулятор, выходной фильтр и компенсацию.

Схема фазы импульсного преобразователя на рис. \ref{tps40140-scheme}:
\addimghere{tps40140_scheme.png}{0.7}{Схема фазы импульсного преобразователя}{tps40140-scheme}

В составе модулятора контроллер TPS40140, управляющий открытием и закрытием транзисторов питания с помощью ШИМ-модуляции, и транзисторы питания. Фильтр включает в себя LC-контур из катушки индуктивности и выходных конденсаторов. Компенсация представляет собой делитель выходного напряжения, подаваемый на вход модулятора.

В одном контроллере TPS40140 реализуется две фазы одного выходного напряжения. Для создания многофазного преобразователя необходимо подключить контроллеры TPS40140 следующим образом:

\addimghere{multiphase1.jpg}{0.35}{Подключение контроллеров в многофазном преобразователе}{multiphase1}

В итоге фазы синхронизируются, и пульсации напряжения минимизируются:
\addimghere{2phase-ripple.png}{0.4}{Пульсации напряжения в многофазном источнике}{2phase-ripple}

Рабочая частота контроллера TPS40140 задается резистором Rtr. Выбрана частота 500кГц:

\begin{equation}
R_{TR} = 1,33 \times (39,2 \times 10^3 \times {f_{PH}}^{-1.058} - 7) = 84,5 \times 10^3
\end{equation}

Пульсации выходного тока определяются значением выходной индуктивности.  

Коэффициент пульсаций тока из-за синхронизации фазы определяется по формуле:

\begin{equation}
I_{RIP\_NORM} = \frac{N_{PH} \times \left(D - \dfrac{m1}{N_{PH}} \right)  \times \left(\dfrac{m1+1}{N_{PH}} - D \right)}{D \times (1-D)} = 0,617
\end{equation}

Индуктивность 
Iripple = 2А

\begin{equation}
L = \frac{(V_{OUT} \times (1-D)) \times I_{RIP\_NORM}}{f_{PH} \times I_{RIPPLE}} = 0,538 \times 10^3
\end{equation}

Минимальная емкость выходных конденсаторов рассчитывается по формуле:

\begin{equation}
Coutmin = I_{TRANMAX}^2 \times \frac{\dfrac{L}{N_{PH}}}{2 \times (V_{IN} - V_{OUT}) \times V_{UNDER}} = 1,92 \text{мФ}
\end{equation}

Установлены конденсаторы общей емкостью 7,28мФ.

Выходное напряжение задается делителем напряжения. Резистор R1=10,7кОм. Требуемое значение Vout=1В.

\begin{equation}
R_{BIAS} = 0,7 \times \frac{R_{1}}{V_{OUT}-0,7} = 25,97\text{кОм}
\end{equation}

Установлен массив резисторов для смены выходного напряжения запаиванием контактных площадок. Значение напряжения по умолчанию 0,997В. На рисунке \ref{vcore-set} показан данный массив резисторов:

\addimghere{vcore_set.png}{0.9}{Задание выходного напряжения питания ядра}{vcore-set} 

Для измерения токов номиналов на входе преобразователей установлены низкоомные резисторы.

\subsubsection{Прочее}

Дополнена шина I2C. К ней подключены следующие компоненты:
\begin{itemize}
	\item КПИ-2 (мастер шины);
	\item МУС (мастер шины);
	\item термодатчики и контроллеры ШИМ вентиляторов LM96163;
	\item преобразователи напряжения General Electrics;
	\item мультиплексор I2C LTC4306;
	\item флеш-память EEPROM FRUID 24FC128-I/ST;
	\item модули памяти;
	\item устройства PCI и PCI Express;
	\item микропроцессор.
\end{itemize}

Карта I2C устройств модуля E8C2-EATX представлена на рисунке \ref{i2c-map}:

\addimghere{i2c-map.png}{0.95}{Карта I2C-устройств модуля E8C2-EATX}{i2c-map}

Разработаны 5 шин JTAG. На шинах установлены следующие компоненты:
\begin{itemize}
	\item шина физуровней Ethernet, устройств PCI Express, PCI;
	\item шина модулей памяти микропроцессора;
	\item шина микропроцессоров;
	\item шина МУС №1;
	\item шина МУС №2.
\end{itemize}

В схеме микросхем флеш-памяти S25FL128SAGNFI001 использован диод BAT54-7-F. На данном диоде на токе в 100мА падение напряжения составляет 0,8В (с 3,3В до 2,3В), что недопустимо для флеш-памяти. Микросхема заменена на BAT750TA с меньшим падением в 0,31В.

В результате проведенной работы составлены электрические схемы модулей E8C2-uATX/SE и E8C2-EATX.

\clearpage
