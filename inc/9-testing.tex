\section{Описание и результаты тестирования. Внедрение}

Произведены модули E8C2-uATX/SE. Модули налажены и протестированы. Модули показали работоспособность всех компонентов и интерфейсов. По результатам тестирования памяти DDR4 модули E8C2-uATX/SE показали лучший результат среди модулей на м.п. "Эльбрус-8С2". На рис. \ref{ddr2400-read} представлена диаграмма, показывающая область работоспособности по значению Vref (ось Y) и сдвига сигнала DQS (ось X) для частоты 2400МГц:
\addimghere{ddr2400-read.png}{0.95}{Диаграмма обрасти работоспособности памяти модуля E8C2-uATX/SE}{ddr2400-read}
Рабочая область частот достигла DDR-2416 (2400МГц), по сдвигу сигнала DQS окно работоспособности в 250Пс больше на 20\%, чем у други модулей на микропроцессоре "Эльбрус-8С2".


Тестирование характеристик памяти модуля E8C2-EATX еще не проведено. 

Во второй версии потребовались следующие доработки:
\begin{itemize}
	\item заменен преобразователь напряжения питания подсистемы памяти на преобразователь с большим максимальным током, с 20А на 40А;
	\item разделено питание разных каналов памяти PWR\_MCx\_VREF\_OUT;
	\item выводы земли на штыревых соединителях USB отключены от CHASSIS;
	\item пины генераторов синхросигнала утянуты к земле;
	\item после исследования задано напряжения питания ядра -- 1,05В, +0V9\_UNCORE -- 1В.
\end{itemize}

Во второй версии проведены данные доработки.

Модуль E8C2-EATX также произведен, налажен и протестирован. В модуле высокая плотность компонентов возле микропроцессора не позволяет устанавливать все типы радиаторов. Модули показали работоспособность всех компонентов и интерфейсов. Данный недостаток будет устранен во второй версии.

Модуль МУС-А произведен, налажен и протестирован. 
Проведены следующие доработки: удален источник напряжения физуровней Ethernet на 1,2В, т.к. в физуровнях имеется встроенный источник. Также от пина сигнала сброса микросхемы памяти отключен сигнал сброса уровня 3,3В. Вывод утянут к питанию уровня 1,5В.

В модуле МУС-А проведены работы по разработке и доработке программных средств. Программные средства предоставляют требуемый функционал.

\clearpage
