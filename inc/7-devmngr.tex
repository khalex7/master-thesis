\section{Разработка модуля управления системой}

В данном разделе представлены этапы разработки модуля МУС.

\subsection{Основные ограничения}

Модуль управления системой МУС-А построен на основе микросхемы Aspeed AST2400, поэтому все интерфейсы, функции и даже микросхемы указаны в документации на микросхему. Требуется выбрать необходимые интерфейсы и сопутствующие микросхемы модуля.

Памятью микросхемы AST2400 выбрана микросхема типа DDR3.

Схематически интерфейсы и микросхемы модуля представлены на функциональной схеме -- рисунке \ref{aspeed-mngr-scheme}:
\addimghere{aspeed-mngr-scheme.png}{0.8}{Схема модуля управления системой}{aspeed-mngr-scheme}

Входные номиналы системы питания модуля - стартовые номиналы платы +5V\_SB и +3V3\_SUS. Было решено получать внутренние номиналы из входного номинала +3V3\_SUS (далее в этой главе +3V3).

Микросхема AST2400 требует входные напряжения:

\begin{itemize}
	\item +3V3 (3,3В, 0,19А);
	\item +1V26 (1,26В, 0,77А);
	\item +1V5 (1,54В, 0,45А), питание подсистемы памяти AST2400 и микросхемы DDR3.
\end{itemize}

Кроме того в модуле применены следующие потребители напряжения: 

\begin{itemize}
	\item 2 микросхемы физуровня RMII, с входным номиналом +3V3, потребляющие ~0,1А;
	\item 3 генератора синхросигнала (+3V3, 0,01-0,03А);
	\item 2 микросхемы физуровня RS-232 MAX3243EIPW (+3V3, 0,06А);
	\item 1 микросхема флеш-памяти SPI (+3V3, 0,1А).
\end{itemize}

Исходя из этих данных для получения номиналов +1V26 и +1V5 требуется установка линейный преобразователей напряжения.

Общая схема системы питания показана на рисунке \ref{mngr-power}
\addimghere{aspeed-mngr_power.png}{0.8}{Схема системы питания МУС}{mngr-power}


\subsection{Подбор компонентов}

Ранее были рассмотрены линейные преобразователи в таблице \ref{linear-power}.
Из них выбран линейный преобразователь MAX8527EUD+.

Микросхема поддерживает физуровни Ethernet стандартов RGMII и RMII. В модуле PPMM-700R, с которым должен быть совместим разрабатываемый МУС, имеется два интерфейса RMII, поэтому выбор среди микросхем физуровня Ethernet следует производить для данного стандарта.

В документации представлены следующие микросхемы PHY RMII, работоспособность которых с данной микросхемой была подтверждена:

\begin{table}[H]
	\caption{Допустимые физуровни RMII}\label{rmii-phy}
	\begin{tabular}{|l|l|l|l|l|l|l|}
		\hline Номер микросхемы & 
		\begin{sideways} Broadcom BCM5221 \end{sideways} & 
		\begin{sideways} Realtek RTL8201E \end{sideways} &
		\begin{sideways} Realtek RTL8201F \end{sideways} &
		\begin{sideways} Intel I210 \end{sideways} & 
		\begin{sideways} Intel I350 \end{sideways} &
		\begin{sideways} Micrel KSZ8021 \end{sideways} \\
		\hline Количество физуровней RMII & 1 & 1 & 1 & 1 & 4 & 1 \\
		\hline Цена, \$ & 4 & 1 & 1,8 & 4 & 10 & 2 \\
		\hline 
	\end{tabular}
\end{table}

Выбрана наиболее дешевая микросхема RTL8201EL-VC-GR. 

Среди микросхем памяти была выбрана микросхема Samsung K4B1G1646G, 128МБ в связи с возможностью скорой поставки. Они может быть заменена микросхемой Samsung K4B1G1646I или аналогичной.

На вход синхросигнала микросхемы AST2400 требуется подавать синхросигнал с частотой 24, 48 или 25 МГц. В случае подачи 25МГц требуется отдельный синхросигнал для интерфейса USB. Выбран Epson SG-8002CA-48.0000-PC-M с частотой 48МГц.

На вход микросхем физуровня RMII требуется подать синхросигнал 50МГц. Выбран генератор синхросигнала KXO-V97T 50.0 MHz.

\newpage
\subsection{Электрическая схема}

При разработке электрической схемы также необходимо воспользоваться руководством по разработке микросхемы AST2400. В нем указаны базовые схемы для микросхемы.

В преобразователях напряжения необходимо произвести расчет сопротивлений обратной связи, задающих выходное напряжение.
В МУС применены преобразователи MAX8527EUD+, рассмотренные ранее.

Vfb=0,5В; 

Номинал +1V26:

R2=2,61кОм, расчет сопротивления для искомого напряжения:

\begin{equation}
R_{1} = R_{2} \times \left(\frac{V_{OUT}}{V_{FB}} - 1 \right) = 3,97\text{кОм}
\end{equation}

Взят резистор R1=4,02кОм; получено выходное напряжение:

\begin{equation}
V_{OUT} = V_{FB} \times \left(1 + \frac{R_{1}}{R_{2}} \right) = 1,27\text{В}
\end{equation}


Номинал +1V54:

R2=2,37кОм, расчет сопротивления для искомого напряжения:

\begin{equation}
R_{1} = R_{2} \times \left(\frac{V_{OUT}}{V_{FB}} - 1 \right) = 4,93\text{кОм}
\end{equation}

Взят резистор R1=4,99кОм; получено выходное напряжение:

\begin{equation}
V_{OUT} = V_{FB} \times \left(1 + \frac{R_{1}}{R_{2}} \right) = 1,55\text{В}
\end{equation}

Микросхема Aspeed AST2400, как показано на рисунке \ref{aspeed-mngr-scheme}, имеет следующие интерфейсы:

\begin{itemize}
	\item интерфейс памяти DDR3, подключен к микросхеме памяти;
	\item синхросигнал микросхемы AST2400;
	\item интерфейс SPI флешки ASPEED, в ней хранится ОПО и СПО МУС;
	\item интерфейсы I2C управления датчиками, ветиляторами и преобразователями напряжения;
	\item интерфейс UART1, вход последовательного порта системы;
	\item интерфейс UART2, выход последовательного порта МУС;
	\item интерфейсы RMII, Ethernet МУС;
	\item интерфейс SPI бута, прошивка микросхемы бута или NVRAM;
	\item SRST\# и SYSRESET\_OUT\#, сингналы сброса с системы и в систему;
	\item GPIO, считывание состояния и управление сигналами системы, поданными на GPIO МУС, такими как включение, выключение.
\end{itemize}

В модуле Pigeon Point PPMM-700R не предусмотрено программирование модуля по интерфейсу SPI. Требовалось реализовать такую возможность на имеющихся интерфейсах на существующих платах. Для програмирования SPI флешки МУС был выбран интерфейс последовательного порта МУС, который на платах предприятия идет напрямую на 10-пиновый штыревой соединитель, а интерфейс последовательного порта можно урезать до меньшего числа линий (вплоть до 3х).

В итоге было выведено 4 сигнала SPI (CS\#, MOSI, MISO, SCK) и 4 сигнала последовательного порта (RXD, TXD, RTS\#, CTS\#). Для изоляции флеш-памяти от возможной подачи сигналов последовательного порта уровня RS-232 (-12В; +12В) был установлен микропереключатель, ключи которого необходимо устанавливать в активное состояние только при программировании МУС. При этом один ключ отключает интерфейс SPI микросхемы AST2400 от флеш-памяти для предотвращения обращений к флешке из разных источников одновременно. Схема интерфейсов представлена на рисунке \ref{aspeed-mngr-scheme}.

\subsection{Размещение и трассировка, проведение DRC, создание КД}

Габариты модуля форм-фактора DDR3 SODIMM 67,6x41,75мм. Толщина регламентируется также стандартом DDR3-SODIMM -- 1мм. Выбрана структура в 8 слоев, из которых 4 сигнальных. 

Были размещены компоненты и произведена трассировка. Расположение основных компонентов на верхнем и нижнем слое показаны на рисунках \ref{mup-as-top} и \ref{mup-as-bottom}.

Размещение компонентов и трассировка на верхнем слое представлена на рисунке \ref{mup-as-top}:
\addimghere{mup-as1.png}{0.70}{Верхний слой МУС-А}{mup-as-top}

Здесь: 1 -- микросхема Aspeed AST2400; 2 -- микросхема памяти DDR3; 3 -- внешний соединитель МУС; 4 -- микропереключатель МУС; 5 -- флешь-память SPI МУС; 6 -- физуровень RMII Eth1 Realtek RTL8201EL. 

Размещение компонентов и трассировка на нижнем слое представлена на рисунке \ref{mup-as-bottom}:
\addimghere{mup-as2.png}{0.70}{Нижний слой МУС-А}{mup-as-bottom}

Здесь: 7 -- физуровень RMII Eth0 Realtek RTL8201EL; 8 -- физуровень RS-232 последовательного порта МУС; 9 -- физуровень RS-232 последовательного порта управляемой системы; 10 -- преобразователи напряжения МУС.

Данный модуль произведен и получен. Проведена наладка модуля, модуль показал свою работоспособность.

Установленный на плате модуль представлен на рисунке \ref{mup-as-in-use}:
\addimghere{mngr1.jpg}{0.70}{МУС-А, установленный на плату}{mup-as-in-use}

Проведено DRC, которое выявило технологические ошибки при разработке. Ошибки исправлены. Также модуль проверен на соответствие документации на микросхему AST2400.

Сгенерированы финальные файлы, которые были предоставлены для создания КД и отправки на завод.

Созданная КД проверена. По ней произведена закупка и монтаж компонентов.

\clearpage
